% Options for packages loaded elsewhere
\PassOptionsToPackage{unicode}{hyperref}
\PassOptionsToPackage{hyphens}{url}
%
\documentclass[
  english,
  man]{apa6}
\usepackage{amsmath,amssymb}
\usepackage{lmodern}
\usepackage{ifxetex,ifluatex}
\ifnum 0\ifxetex 1\fi\ifluatex 1\fi=0 % if pdftex
  \usepackage[T1]{fontenc}
  \usepackage[utf8]{inputenc}
  \usepackage{textcomp} % provide euro and other symbols
\else % if luatex or xetex
  \usepackage{unicode-math}
  \defaultfontfeatures{Scale=MatchLowercase}
  \defaultfontfeatures[\rmfamily]{Ligatures=TeX,Scale=1}
\fi
% Use upquote if available, for straight quotes in verbatim environments
\IfFileExists{upquote.sty}{\usepackage{upquote}}{}
\IfFileExists{microtype.sty}{% use microtype if available
  \usepackage[]{microtype}
  \UseMicrotypeSet[protrusion]{basicmath} % disable protrusion for tt fonts
}{}
\makeatletter
\@ifundefined{KOMAClassName}{% if non-KOMA class
  \IfFileExists{parskip.sty}{%
    \usepackage{parskip}
  }{% else
    \setlength{\parindent}{0pt}
    \setlength{\parskip}{6pt plus 2pt minus 1pt}}
}{% if KOMA class
  \KOMAoptions{parskip=half}}
\makeatother
\usepackage{xcolor}
\IfFileExists{xurl.sty}{\usepackage{xurl}}{} % add URL line breaks if available
\IfFileExists{bookmark.sty}{\usepackage{bookmark}}{\usepackage{hyperref}}
\hypersetup{
  pdflang={en-EN},
  hidelinks,
  pdfcreator={LaTeX via pandoc}}
\urlstyle{same} % disable monospaced font for URLs
\usepackage{graphicx}
\makeatletter
\def\maxwidth{\ifdim\Gin@nat@width>\linewidth\linewidth\else\Gin@nat@width\fi}
\def\maxheight{\ifdim\Gin@nat@height>\textheight\textheight\else\Gin@nat@height\fi}
\makeatother
% Scale images if necessary, so that they will not overflow the page
% margins by default, and it is still possible to overwrite the defaults
% using explicit options in \includegraphics[width, height, ...]{}
\setkeys{Gin}{width=\maxwidth,height=\maxheight,keepaspectratio}
% Set default figure placement to htbp
\makeatletter
\def\fps@figure{htbp}
\makeatother
\setlength{\emergencystretch}{3em} % prevent overfull lines
\providecommand{\tightlist}{%
  \setlength{\itemsep}{0pt}\setlength{\parskip}{0pt}}
\setcounter{secnumdepth}{-\maxdimen} % remove section numbering
% Make \paragraph and \subparagraph free-standing
\ifx\paragraph\undefined\else
  \let\oldparagraph\paragraph
  \renewcommand{\paragraph}[1]{\oldparagraph{#1}\mbox{}}
\fi
\ifx\subparagraph\undefined\else
  \let\oldsubparagraph\subparagraph
  \renewcommand{\subparagraph}[1]{\oldsubparagraph{#1}\mbox{}}
\fi
% Manuscript styling
\usepackage{upgreek}
\captionsetup{font=singlespacing,justification=justified}

% Table formatting
\usepackage{longtable}
\usepackage{lscape}
% \usepackage[counterclockwise]{rotating}   % Landscape page setup for large tables
\usepackage{multirow}		% Table styling
\usepackage{tabularx}		% Control Column width
\usepackage[flushleft]{threeparttable}	% Allows for three part tables with a specified notes section
\usepackage{threeparttablex}            % Lets threeparttable work with longtable

% Create new environments so endfloat can handle them
% \newenvironment{ltable}
%   {\begin{landscape}\begin{center}\begin{threeparttable}}
%   {\end{threeparttable}\end{center}\end{landscape}}
\newenvironment{lltable}{\begin{landscape}\begin{center}\begin{ThreePartTable}}{\end{ThreePartTable}\end{center}\end{landscape}}

% Enables adjusting longtable caption width to table width
% Solution found at http://golatex.de/longtable-mit-caption-so-breit-wie-die-tabelle-t15767.html
\makeatletter
\newcommand\LastLTentrywidth{1em}
\newlength\longtablewidth
\setlength{\longtablewidth}{1in}
\newcommand{\getlongtablewidth}{\begingroup \ifcsname LT@\roman{LT@tables}\endcsname \global\longtablewidth=0pt \renewcommand{\LT@entry}[2]{\global\advance\longtablewidth by ##2\relax\gdef\LastLTentrywidth{##2}}\@nameuse{LT@\roman{LT@tables}} \fi \endgroup}

% \setlength{\parindent}{0.5in}
% \setlength{\parskip}{0pt plus 0pt minus 0pt}

% Overwrite redefinition of paragraph and subparagraph by the default LaTeX template
% See https://github.com/crsh/papaja/issues/292
\makeatletter
\renewcommand{\paragraph}{\@startsection{paragraph}{4}{\parindent}%
  {0\baselineskip \@plus 0.2ex \@minus 0.2ex}%
  {-1em}%
  {\normalfont\normalsize\bfseries\itshape\typesectitle}}

\renewcommand{\subparagraph}[1]{\@startsection{subparagraph}{5}{1em}%
  {0\baselineskip \@plus 0.2ex \@minus 0.2ex}%
  {-\z@\relax}%
  {\normalfont\normalsize\itshape\hspace{\parindent}{#1}\textit{\addperi}}{\relax}}
\makeatother

% \usepackage{etoolbox}
\makeatletter
\patchcmd{\HyOrg@maketitle}
  {\section{\normalfont\normalsize\abstractname}}
  {\section*{\normalfont\normalsize\abstractname}}
  {}{\typeout{Failed to patch abstract.}}
\patchcmd{\HyOrg@maketitle}
  {\section{\protect\normalfont{\@title}}}
  {\section*{\protect\normalfont{\@title}}}
  {}{\typeout{Failed to patch title.}}
\makeatother
\shorttitle{SHORTTITLE}
\usepackage{csquotes}
\ifxetex
  % Load polyglossia as late as possible: uses bidi with RTL langages (e.g. Hebrew, Arabic)
  \usepackage{polyglossia}
  \setmainlanguage[]{english}
\else
  \usepackage[main=english]{babel}
% get rid of language-specific shorthands (see #6817):
\let\LanguageShortHands\languageshorthands
\def\languageshorthands#1{}
\fi
\ifluatex
  \usepackage{selnolig}  % disable illegal ligatures
\fi

\title{TITLE}
\author{\phantom{0}}
\date{}


\affiliation{\phantom{0}}

\begin{document}
\maketitle

This is the manuscript.


\clearpage



\begin{appendix}
\hypertarget{stimuli}{%
\section{Stimuli}\label{stimuli}}

\hypertarget{words}{%
\subsection{Words:}\label{words}}

\begin{verbatim}
##   [1] GLANCE WINTER MILDEW SLEAZY BOTHER PORTAL COURSE SOCIAL CENSOR BACKUP
##  [11] STABLE PLAGUE OBJECT ABSENT RESIGN CREDIT STREAM WICKED INVOKE BEACON
##  [21] ADMIRE FACTOR PHRASE BURDEN LOUNGE TAILOR WREATH HOCKEY SEARCH MONKEY
##  [31] INFECT VERIFY SQUEAL INTAKE PHOBIA BEHALF LIQUOR SHRINK BUCKET VOYAGE
##  [41] MASCOT HINDER KIDNEY PERIOD VULGAR CASINO SELDOM MASTER MOTHER FRINGE
##  [51] EXPORT INJECT NATURE CLUMSY GRUMPY PIRATE POSTER FLOWER LUMBER FOREST
##  [61] FRIGHT BRIDGE KNIGHT TRENCH PLUNGE BUNDLE ANCHOR ABRUPT HUSTLE ETHNIC
##  [71] CRUISE ORANGE STRIVE ALMOST AMOUNT FAUCET CHARGE HUMBLE PATRON BREAST
##  [81] SIMPLE JOCKEY WEALTH PRAISE DOMAIN SPRINT SCREAM COMEDY SKETCH ORIENT
##  [91] WEAPON BUCKLE BUTLER JUNGLE SOCKET VANISH INSULT POWDER INSECT CONVEY
## [101] SPRING CHAPEL TONGUE DANGER CLERGY SINGLE INCOME WALNUT SURVEY SNATCH
## [111] HUNGRY IMPOSE POLICE RAMBLE STROKE AMBUSH CHANGE FOSTER SPLINT PENCIL
## [121] PONDER REGION LAUNCH THRONE SYMBOL GOSPEL POLICY FUMBLE PLENTY SQUARE
## [131] GARDEN BRIGHT SOURCE PLAQUE TROPHY EXOTIC RACKET COLUMN THEORY FAMILY
## [141] FRIEND STIGMA RESULT MORTAL SPRAWL RODENT FILTER HUNGER PERMIT RELISH
## [151] EXPAND STRAND PUBLIC TRAVEL THREAD IGNORE CLIENT BELONG FLUENT CARBON
## [161] WRENCH DIRECT BOUNCE STUDIO NUMBER RANSOM SLOGAN RECKON WONDER POLITE
## [171] INDUCE BREATH CHORUS PLACID GUITAR TUMBLE BRANCH FLIGHT SPIDER STRAIN
## [181] THRIVE LINGER STRIKE BLOUSE NICKEL ENOUGH CASTLE SENIOR MARVEL STUPID
## [191] QUENCH STRONG CHROME STAPLE IMPORT BLEACH FINGER MELODY DEBRIS PRINCE
## [201] ORPHAN SPRUCE REASON SPONGE PARISH COUPLE GARLIC CUSTOM INVADE RANDOM
## [211] MYSTIC ROCKET BEHOLD AUTHOR LIZARD CRADLE DRAGON ISLAND DETAIL VIOLET
## [221] PASTOR CANDLE STRING MARKET INVEST STARCH DESIGN ANSWER GENIUS POUNCE
## [231] DEPART PATROL SHOWER TURKEY FABRIC STRIFE ADVENT INFORM BASKET SILENT
\end{verbatim}

\hypertarget{non-words}{%
\subsection{Non-Words:}\label{non-words}}

\begin{verbatim}
##   [1] KNISMA ALCOUD STRELD PRAILE FRUDGE URIVEL SCHIND TANDLY FUNTLE CAROEY
##  [11] CHRUMS LUNKER DRANCE PITROM IMPURB FRUTAN SNAPEL BAITCH LIMPOR SARIDE
##  [21] RASTOE ABOUNE DORSEX JAUNCE MINKEL RUNAMO GIRCUE SADENT BRUNGE PHRECT
##  [31] CAINKY SURDEN BOLVIS SCOUGE TONDLE BROGET WANIGH BARCET BAFENT MINGAR
##  [41] JUSTRE ABSULT KISPEL CHAITE WHIVEX HOCITE GORBIE WICTOR PADIFS CRUDIO
##  [51] NOBEST CIGNEY CRIBLE HABLIN PEBRIC JOSTED PHOTIE ACRISE LOBUSH ROBULY
##  [61] SMENCO UNWERT MUSIED EATRIC ANOUGS PIATON BLARGE STROLE BATROL MILTED
##  [71] LUPICT STAPOD RESAIT VAROSH APIGHT INLORS VENIOD POLUTH SOUDAL UNJECT
##  [81] STUNCH QUILEW DOLICA VACKEL DISTAR VARMIT PASINK PACKES ORCHET WRIMSY
##  [91] AUBRID DAMILT SORTEL FUMBLO SLEIRT YARMON PETAIN UNIGHS CELOND CURBOL
## [101] MELATY PORLEX VERALF SHAPLE PRUDGE RELDON NYMBEL YACEUP STARBY SPLINO
## [111] PRETCH ABLINT SWANCE INHORT GLEATH BEINCH FERTIC PIERGY STURCH YARION
## [121] FAINCH MILUER FIREAM CLUDIE FORGLE SOUNGE KIRAFE PSETCH AMILOY DANIET
## [131] SPROLY NERAWL PAURSE SPRINE LASTON SCREGM WORBAL REAGLY ZOMBER JOCKAL
## [141] MIATOR SLANCH ROSAIL STEIKH THRUSE IGUADE PLENAC BROAKE YARIKE KENSOM
## [151] GILFEW VANGUE CATRIE WATMEG DESION GHODUS REJOLS WREASH FACHOW FIATCH
## [161] MYRTIE RANDOW AIMOSY SCRILY ABSEND SAUDIC KERIFT DUMPLE ASYLEN NAGURS
## [171] TIESCH TRINGE CLUREY VIRLEX ZINGLY OTHNIS DERAIN FLEACK FONVER DESORY
## [181] PAUCHE TIREAU BREALI IMPOTE GUAIRT TOSHEL SLIQUE HORNAL WRONCH SLEACT
## [191] GEILOY RAMILY BIGENT SLIANT YORQUE IGUARE INJORE SHRILE AUTING CAVORY
## [201] HUNIEK SARLEY AUNGRE GUESCO VEIGMA ANCUST GRONCE CRAGIN WITMER HATRUS
## [211] THELDY ROUNCY GANOUS SPAQUX COLURY FLENGY CAMBLE BLUNDE FOURET WEAROX
## [221] MUSTIL SYNTIC BRUATS CLUIRM FLUMNI NOUPLY SUIVER HURNIA ELOPIA TUNORY
## [231] ORCHIN JILUER EQUATS SERBOL HIGSTA PATHEL EAROUD SUIDLE SLOUNT GUSHOP
\end{verbatim}
\end{appendix}

\clearpage



\begin{appendix}
\hypertarget{diffusion-model-accounts-for-the-data}{%
\section{Diffusion model accounts for the
data}\label{diffusion-model-accounts-for-the-data}}

\hypertarget{about-this-appendix}{%
\subsection{About this Appendix}\label{about-this-appendix}}

The goal of this appendix is to present a brief description of how the
diffusion model accounts for the data presented in the main article.
While the diffusion model fits to the grouped data was part of the
pre-registration plan, we believe that it is best to present such fits
in this appendix as opposed to the main text to improve the readability
of the article. For a full description of the experimental setting and
its goals please refer to the main text.

\hypertarget{the-diffusion-model}{%
\subsection{The diffusion model}\label{the-diffusion-model}}

The diffusion model (Ratcliff, 1978) is a cognitive process model for
perceptual decisions, and it has been quite successful at accounting for
lexical decision data (Ratcliff et al, 2004) and more importantly for
the present work, masked priming data (Gomez et al, 2012).

The model assumes that RTs to dual choice tasks are a sum of three
distinct processes: stimulus encoding, evidence accumulation, and
response execution. The model makes the strong assumption that evidence
accumulation is a processes distinct from the other two components, and
for practical reasons, it groups response execution and encoding time in
a single parameter.

The model is agnostic about the correlation between the encoding and the
evidence accumulation processes; and we like to think about it as a tool
to instantiate theoretical positions that can be articulated in terms of
encoding, decision, and strategic processes.

\hypertarget{data}{%
\subsection{Data}\label{data}}

The model was fit to the grouped data (as per the pre-registration plan)
of the two experiments. For each stimulus type the proportion of word
and nonword responses is calculated, then for each of the two responses,
the RTs at the .1, .3, .5, .7, and .9 quantiles is obtained. We repeat
this process for each participants, and then all of those quantities
(response proportions and RTs at quantiles) are averaged across
participants. This process is also known as Vincentalizing, and the
averaged quantiles are referred to as vintenciles

The diffusion model predicts the cumulative probability of a response at
each RT vincentile, and these model predictions are compared to the
empirical proportions, then the sum of the
(Observed-Predicted)2/Predicted for correct and error responses for each
condition that is minimized with a general SIMPLEX minimization routine
as described by Ratcliff \& Tuerlinckx (2002).

\hypertarget{free-and-fixed-parameters}{%
\subsection{Free and fixed parameters}\label{free-and-fixed-parameters}}

In diffusion model fits, researchers can decide what parameters are free
to vary across different conditions. In the present work we implemented
3 versions of the model. These versions of the model varied in terms of
which parameters were allowed to vary for which conditions.

In our case, we decided to examine two models as described below. For
both models, the \(a\) boundary separation, the \(z\) starting point,
the \(\eta\) between trial variability in drift rate, and all other
variability parameters are kept constant across all conditions.

\textbf{Model 1:} Drift rates vary as a function of lexicality and prime
duration, but not from unrelated to identity primes. \(T_{er}\) varied
from as a function of prime duration and type and also of lexicality.

\textbf{Model 2:} The drift rates vary as a function of prime duration
and type and also of lexicality. And the drift rate varied only as a
function of prime type/duration but not lexicality.

In short, in Model 1, the priming effects are accounted by \(T_{er}\)
only, while in Model 2 they are accounted for by both drift rate and
\(T_{er}\).

The two models have equal number of parameters so a direct comparison in
possible. For both experiments the preferred model is Model 1 (the
\(T_{er}\) model). This is in agreement with the Gomez et al (2012)
study using in-person testing methods.

\begin{table}[tbp]

\begin{center}
\begin{threeparttable}

\caption{\label{tab:model_parameters_table}$\chi^2$ values for the two models for both experiments.}

\begin{tabular}{lll}
\toprule
Experiments & \multicolumn{1}{c}{ModelTer} & \multicolumn{1}{c}{ModelDrift}\\
\midrule
1 & 39.65 & 80.65\\
2 & 10.21 & 28.65\\
\bottomrule
\end{tabular}

\end{threeparttable}
\end{center}

\end{table}

\hypertarget{summary}{%
\subsection{Summary}\label{summary}}

Examining the parameter values for \(T_{er}\) in the tables below shows
that the \(T_{er}\) effect follows the duration of the prime-TARGET SOA
particularly in the word items and not so much in the nonword items.

This is in general agreement with the Gomez et al.~(2013) original paper
and Gomez and Perea's (2020) work with developmental readers. In short,
these fits confirm that masked priming effects are consistent with the
idea of a head start in the encoding process when there is an identity
relationship between primes and targets.

\begin{table}[tbp]

\begin{center}
\begin{threeparttable}

\caption{\label{tab:appendix_table_2}$T_{er}$ values for Experiment 1.}

\begin{tabular}{ll}
\toprule
Parameters & \multicolumn{1}{c}{Experiment1}\\
\midrule
$T_{er}$ values for Experiment 1.  33ms nonword ID & 0.51\\
$T_{er}$ values for Experiment 1. 33ms nonword Unrel & 0.51\\
$T_{er}$ values for Experiment 1.  50ms nonword ID & 0.50\\
$T_{er}$ values for Experiment 1. 50ms nonword Unrel & 0.51\\
$T_{er}$ values for Experiment 1.  33ms word ID & 0.47\\
$T_{er}$ values for Experiment 1. 33ms word Unrel & 0.49\\
$T_{er}$ values for Experiment 1.  50ms word ID & 0.46\\
$T_{er}$ values for Experiment 1. 50ms word Unrel & 0.50\\
\bottomrule
\end{tabular}

\end{threeparttable}
\end{center}

\end{table}

\begin{table}[tbp]

\begin{center}
\begin{threeparttable}

\caption{\label{tab:appendix_table_3} \$T\_\{er\}\$ values for Experiment 2.}

\begin{tabular}{ll}
\toprule
Parameters & \multicolumn{1}{c}{Experiment2}\\
\midrule
\$T\_\{er\}\$ 16ms nonword ID & 0.54\\
\$T\_\{er\}\$ 16ms nonword Unrel & 0.54\\
\$T\_\{er\}\$ 33ms nonword ID & 0.53\\
\$T\_\{er\}\$ 33ms nonword Unrel & 0.53\\
\$T\_\{er\}\$ 16ms word ID & 0.50\\
\$T\_\{er\}\$ 16ms word Unrel & 0.50\\
\$T\_\{er\}\$ 33ms word ID & 0.49\\
\$T\_\{er\}\$ 33ms word Unrel & 0.51\\
\bottomrule
\end{tabular}

\end{threeparttable}
\end{center}

\end{table}

\newpage

\hypertarget{references}{%
\subsection{References}\label{references}}

Gomez, P., \& Perea, M. (2020). Masked identity priming reflects an
encoding advantage in developing readers. Journal of Experimental Child
Psychology, 199, 104911. https://doi.org/10.1016/j.jecp.2020.104911

Gomez, P., Perea, M., \& Ratcliff, R. (2013). A diffusion model account
of masked versus unmasked priming: Are they qualitatively different?
Journal of Experimental Psychology: Human Perception and Performance, 39
(6), 1731--1740. https://doi.org/10.1037/a0032333

Ratcliff, R. (1978). A theory of memory retrieval. Psychological Review,
85, 59--108. https://doi.org/10.1037/0033-295X.85.2.59

Ratcliff, R., Gomez, P., \& McKoon, G. (2004). A diffusion model account
of the lexical decision task. Psychological Review, 111, 159--182.
doi://10.1037/0033-295X.111.1.159

Ratcliff, R.,\& Tuerlinckx, F. (2002). Estimating the parameters of the
diffusion model: Approaches to dealing with contaminant reaction times
and parameter variability. Psychonomic Bulletin and Review, 9, 438--481.
https://doi.org/10.3758/BF03196302
\end{appendix}

\end{document}
